\begin{titlepage}
    { % 使用内部块
        \centering % 整个页面居中    
        {\bfseries\Large 深圳大学研究生课程期末论文评分表 \\[0.8cm]}
        \raggedright                % 段落从左对齐,避免段落居中影响正文

        \zihao{-4} % 设置正文字体为小四号

        \noindent
        课程名称:\underline{\hspace{4cm}课程的名称\hspace{4.4cm}}

        \noindent
        论文题目:\underline{\hspace{1cm}请输入论文标题(通过hspace控制横线长度)\hspace{1cm}}

        \noindent
        学\hspace{2em}号:\underline{\hspace{1cm}0123456789\hspace{1cm}} \hspace{1em} 姓\hspace{2em}名:\underline{\hspace{3em}名字\hspace{3em}}

        \vspace{0.5cm} % 表格下方留出 0.5cm 的空白

        \raggedright                % 段落从左对齐,避免段落居中影响正文
        \setstretch{1.25}            % 设置下面段落行间距为 1.25 倍
        \setlength{\parskip}{0pt}   % 段后间距设为 0
        % \renewcommand{\arraystretch}{1.8} % 行距增大,视觉美观
        % \setlength{\tabcolsep}{3pt}       % 列间距调整
        % 使用 tabularx 表格
        % 表格部分
        \renewcommand{\arraystretch}{1.7}   % 行距增大,视觉美观
        \begin{tabularx}{\textwidth}{|>{\centering\arraybackslash}m{5em}|X|>{\centering\arraybackslash}m{4em}|>{\centering\arraybackslash}m{4em}|}
            \hline  % 划线
            \textbf{指标} & \textbf{评分标准} & \textbf{分值} & \textbf{得分} \\
            \hline
            文献 & 文献资料是否恰当、详实;是否具有代表性;是否有述有评。 & 10 &  \\
            \hline
            选题 & 选题是否新颖;是否有理论意义或实用价值;是否与授课内容相符。 & 10 &  \\
            \hline
            规范 & 篇幅字数在规定要求范围内;文字表达是否准确、流畅;论述是否具有论辩性;图表计量单位是否规范;是否符合学术道德规范,论文独立完成,无抄袭现象。 & 30 &  \\
            \hline
            论证 & 研究方案是否具有可行性;是否能较好运用所学知识,观点明确;思路是否清晰;逻辑是否严密;结构是否严谨;论证是否充分。 & 30 &  \\
            \hline
            实用价值 & 调研成果是否具有实际应用价值;是否提出了可行的建议或解决方案;是否对相关领域有参考意义;是否体现创新思维。 & 20 &  \\
            \hline
            其他意见(选填) & \multicolumn{3}{l|}{} \\ % 不使用'\rule{10cm}{0pt}',这样以防表格最右端的线不贴合表格,出现短断裂(如下一行所示,可以取消下一行的注释对比效果)
            % 其他意见(选填) & \multicolumn{3}{l|}{\rule{10cm}{0pt}} \\
            \hline
            \multicolumn{2}{|l|}{\textbf{任课教师签名:}\rule{7cm}{0pt}} & \multicolumn{2}{l|}{\textbf{总分:}} \\  % 不使用'\rule{3cm}{0pt}',这样以防表格最右端的线不贴合表格,出现短断裂(如下一行所示,可以取消下一行的注释对比效果)
            % \multicolumn{2}{|l|}{任课教师签名:\rule{7cm}{0pt}} & \multicolumn{2}{l|}{总分:\rule{3cm}{0pt}} \\
            % \hline  % 不使用\hline实线,使两行在PDF中紧密相连,从视觉上变成一行
            \multicolumn{2}{|l|}{\hspace{13em}年\underline{\hspace{1cm}}月\underline{\hspace{1cm}}日} & \multicolumn{2}{l|}{}\\
            \hline
        \end{tabularx}

        \vspace{0.5cm} % 表格下方留出 0.5cm 的空白
        \small
        \noindent
        1. 该表应在期末考试前由任课教师发给学生,告知学生论文评分标准;\\
        2. 学生应在提交期末论文时,封面附上该表并补充填写好表格基本个人信息。
    }
\end{titlepage}