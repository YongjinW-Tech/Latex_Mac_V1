% 学术诚信承诺书页
\begin{titlepage}
    { % 使用内部块
        \centering % 整个页面居中

        {\bfseries \Large 深圳大学研究生课程论文学术诚信承诺书 \\[1.5cm]}
        \raggedright                % 段落从左对齐,避免段落居中影响正文
        \setstretch{1.8}            % 设置下面段落行间距为 1.8 倍
        \setlength{\parskip}{0pt}   % 段后间距设为 0
        \setlength{\parindent}{2em} % 首行缩进 2 个字符

        \zihao{4} % 设置正文字体为四号
        
        本人在此声明所提交的课程论文 \underline{\hspace{6em}}(论文标题)是本人独立完成的,具有原创性,并且未抄袭、剽窃他人成果或侵犯他人的知识产权。本声明书详细阐述以下内容:

        1. 本人郑重声明,课程论文的所有内容和观点均源自本人的研究和分析,未从其他来源直接复制或翻译。

        2. 对于其他作者或研究人员的观点、数据、图片、图表等引用和参考,本人已按照学校规定的引用标准进行准确的引用和注明,并在文中明确标明了引用部分。

        3. 本人保证,课程论文中使用的所有文献、资料和其他来源均已在参考文献部分列出,且准确无误地注明了相关信息,包括作者、出版年份、出版社或期刊名称等。

        4. 本人明确知晓学术不端行为的严重性,包括但不限于抄袭、剽窃、造假、篡改数据等。本人承诺,在课程论文的整个研究和撰写过程中,坚守学术道德原则,维护学术诚信。

        我郑重承诺以上内容的真实性,并愿意为我所提交的课程论文的原创性负全部责任。

        \vspace{1cm}
        % \zihao{-4} % 设置签名部分字体为小四号
        \setlength{\parindent}{0pt} % 下一行段首不缩进
        
        %%  下面的签名行可以使用图片或手写签名
        % 下面的图片需要你自己准备,放在 Imgs 文件夹下,并命名为 Electronic-Signature.jpg
        % 论文作者签名:\raisebox{-0.5\height}{\includegraphics[width=4cm]{Imgs/Electronic-Signature.jpg}} \hfill 日期:\underline{\hspace{0.3cm}2025\hspace{0.3cm}}年\underline{\hspace{0.3cm}06\hspace{0.3cm}}月\underline{\hspace{0.3cm}12\hspace{0.3cm}}日 % \raisebox{-0.5\height}{...} 表示把图片整体往上提半个图片高度,从而大致与前面的文本基线对齐;这样文字与图片看起来是在一条水平线上;这一种做法在签名行、行内小图标等场景非常常用
        
        % 如果需要手写签名,可以使用 \underline{\hspace{4cm}} 代替图片
        论文作者签名:\underline{\hspace{8em}} \hfill 日期:\underline{\hspace{0.3cm}2025\hspace{0.3cm}}年\underline{\hspace{0.3cm}06\hspace{0.3cm}}月\underline{\hspace{0.3cm}12\hspace{0.3cm}}日
    }
\end{titlepage}

%% LaTeX 字体命令参考字体
% LaTeX 实际字号会依赖于 documentclass 里设定的基本字号(比如 10pt、11pt、12pt)
% 下面我以常用的 12pt 版式为例给你做标准对照,符合国内论文排版习惯,方便在论文中选用:
% LaTeX 命令     大致字号     中文习惯称呼     常用场景
% \Huge         24pt        小初(接近)     论文封面、论文题目
% \huge         20pt        小一(接近)     正文一级大标题
% \LARGE       17pt        二号(接近)     封面副标题、扉页标题
% \Large       14pt        三号(接近)     正文大标题、小节标题
% \large       12pt        小四(正文字号) 正文内适当放大
% \normalsize  12pt        小四             正文默认
% \small       11pt        五号偏小         脚注、附表说明
% \footnotesize 10pt        五号             脚注、参考文献
% \scriptsize  8pt         小五             表格中小字、附注
% \tiny        6pt         小六             非常小,几乎不用


%% 补充国内常用 \zihao{} 对照(学位论文最常用)
%| \zihao{0} | 42pt | 初号 |
%| \zihao{-0} | 36pt | 小初号 |
%| \zihao{1} | 26pt | 一号 |
%| \zihao{2} | 22pt | 二号 |
%| \zihao{3} | 16pt | 三号 |
%| \zihao{4} | 14pt | 四号(正文字号推荐) |
%| \zihao{5} | 10.5pt | 五号 |